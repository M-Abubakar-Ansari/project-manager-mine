\documentclass{article}
\usepackage{amsmath, amssymb}
\usepackage{geometry}
\geometry{a4paper, margin=1in}
\usepackage{parskip}

% Setting up reliable fonts
\usepackage{mathptmx} % Times-based font for compatibility
\usepackage[T1]{fontenc}

\begin{document}

\title{A Mathematical Framework for Impowers: A Division-Centric Perspective on Exponentiation}
\author{}
\date{October 2025}
\maketitle

\begin{abstract}
This paper introduces and explores the concept of ``impowers,'' a division-centric reparameterization of exponentiation defined as \(\text{Im}_n(a) = a^{1-n}\) for a scalar \(a \neq 0\). Originally conceived to formalize repeated division, impowers provide an intuitive framework for studying iterative inverse processes. We define the general form, derive key algebraic rules, theorems, and corollaries, investigate their behavior in polynomial equations, and analyze sequences and series generated by impowers. Special attention is given to complex numbers (e.g., \(a = i\)) to highlight physical significance in oscillatory systems. While impowers are a redefinition of exponentiation, their notation and perspective offer pedagogical and applied value, particularly in signal processing and quantum mechanics.
\end{abstract}

\section{Introduction}
Exponentiation formalizes repeated multiplication, yielding structures like geometric sequences and polynomial equations. However, repeated division, its inverse operation, is less explicitly structured in standard mathematics. The concept of ``impowers,'' introduced by the author, defines \(\text{Im}_n(a) = a^{1-n}\), encapsulating the result of dividing \(a\) by itself \(n-1\) times. This paper formalizes impowers, derives their algebraic properties, and explores their behavior in equations and sequences. We emphasize applications for complex numbers, where impowers model phase shifts in physical systems.

\section{Definition}
For a non-zero scalar \(a \in \mathbb{C}\), the \(n\)-th impower is defined as:
\[
\text{Im}_n(a) = a^{1-n} = \frac{a}{a^{n-1}}, \quad n \in \mathbb{R}.
\]
This represents dividing \(a\) by itself \(n-1\) times. For integer \(n\):
\begin{itemize}
    \item \(\text{Im}_1(a) = a^0 = 1\),
    \item \(\text{Im}_2(a) = a^{-1} = \frac{1}{a}\),
    \item \(\text{Im}_3(a) = a^{-2} = \frac{1}{a^2}\).
\end{itemize}
For complex \(a\), the principal branch of the exponent is used unless specified.

\section{General Form}
The general form of an impower is:
\[
\text{Im}_n(a) = a^{1-n},
\]
where \(n\) can be any real number, extending to continuous indices. This can be rewritten as:
\[
\text{Im}_n(a) = a \cdot a^{-(n-1)} = \frac{1}{a^{n-1}}.
\]
For a fixed \(a\), the function \(\text{Im}_n(a)\) maps \(n\) to a power of \(a\), emphasizing division over multiplication. For complex \(a = re^{i\theta}\), we have:
\[
\text{Im}_n(a) = r^{1-n} e^{i\theta (1-n)}.
\]

\section{Rules, Theorems, and Corollaries}
Impowers obey algebraic rules derived from exponentiation, reframed to highlight division. Below are key theorems and corollaries, with proofs.

\begin{theorem}[Multiplication Rule]
For any \(p, q \in \mathbb{R}\),
\[
\text{Im}_p(a) \cdot \text{Im}_q(a) = \text{Im}_{1+p+q}(a).
\]
\end{theorem}
\begin{proof}
\[
\text{Im}_p(a) \cdot \text{Im}_q(a) = a^{1-p} \cdot a^{1-q} = a^{(1-p) + (1-q)} = a^{2-(p+q)} = \text{Im}_{1+p+q}(a).
\]
\end{proof}
\begin{corollary}
\[
\text{Im}_p(a) \cdot \text{Im}_q(a) = \text{Im}_{p+q}(a) \cdot a.
\]
\end{corollary}
\begin{proof}
\[
a^{2-(p+q)} = a^{1-(p+q)} \cdot a = \text{Im}_{p+q}(a) \cdot a.
\]
\end{proof}

\begin{theorem}[Division Rule]
\[
\frac{\text{Im}_p(a)}{\text{Im}_q(a)} = \text{Im}_{p-q+1}(a).
\]
\end{theorem}
\begin{proof}
\[
\frac{\text{Im}_p(a)}{\text{Im}_q(a)} = \frac{a^{1-p}}{a^{1-q}} = a^{(1-p) - (1-q)} = a^{q-p} = a^{1-(p-q+1)} = \text{Im}_{p-q+1}(a).
\]
\end{proof}

\begin{theorem}[Composition Rule]
\[
\text{Im}_m(\text{Im}_n(a)) = \text{Im}_{m+n-mn}(a).
\]
\end{theorem}
\begin{proof}
\[
\text{Im}_n(a) = a^{1-n}, \quad \text{Im}_m(\text{Im}_n(a)) = (a^{1-n})^{1-m} = a^{(1-n)(1-m)} = a^{1-m-n+mn} = \text{Im}_{m+n-mn}(a).
\]
\end{proof}

\begin{theorem}[Root Rule]
For \(g \neq 0\),
\[
\sqrt[g]{\text{Im}_n(a)} = \text{Im}_n(\sqrt[g]{a}).
\]
\end{theorem}
\begin{proof}
\[
\sqrt[g]{\text{Im}_n(a)} = (a^{1-n})^{1/g} = a^{(1-n)/g} = (a^{1/g})^{1-n} = \text{Im}_n(a^{1/g}).
\]
\end{proof}

\begin{theorem}[Reciprocal Rule]
\[
\text{Im}_n\left(\frac{1}{a}\right) = \frac{1}{\text{Im}_n(a)} = \text{Im}_{2-n}(a).
\]
\end{theorem}
\begin{proof}
\[
\text{Im}_n\left(\frac{1}{a}\right) = \left(\frac{1}{a}\right)^{1-n} = a^{-(1-n)} = a^{n-1}.
\]
Since \(\frac{1}{\text{Im}_n(a)} = \frac{1}{a^{1-n}} = a^{n-1}\), and:
\[
a^{n-1} = a^{1-(2-n)} = \text{Im}_{2-n}(a).
\]
\end{proof}

\begin{theorem}[Derivative Rule]
For continuous \(n\),
\[
\frac{d}{dn} \text{Im}_n(a) = -\text{Im}_n(a) \ln a.
\]
\end{theorem}
\begin{proof}
\[
\text{Im}_n(a) = a^{1-n} = e^{(1-n) \ln a}, \quad \frac{d}{dn} = e^{(1-n) \ln a} \cdot (-\ln a) = -\text{Im}_n(a) \ln a.
\]
\end{proof}

\section{Equations Involving Impowers}
Impowers can appear in polynomial equations, with solutions for the index \(n\).

\subsection{Quadratic Equations}
Consider:
\[
(\text{Im}_n(a))^2 + b \text{Im}_n(a) + c = 0.
\]
Substitute \(\text{Im}_n(a) = a^{1-n}\):
\[
a^{2-2n} + b a^{1-n} + c = 0.
\]
Multiply by \(a^{2n-2}\):
\[
1 + b a^{n-1} + c a^{2n-2} = 0.
\]
Let \(y = a^{n-1}\):
\[
c y^2 + b y + 1 = 0, \quad y = \frac{-b \pm \sqrt{b^2 - 4c}}{2c}.
\]
Thus:
\[
n = 1 + \log_a \left( \frac{-b \pm \sqrt{b^2 - 4c}}{2c} \right).
\]
For \(a = i\), \(b = 1\), \(c = 1\):
\[
y^2 + y + 1 = 0, \quad y = \omega, \omega^2, \quad \omega = e^{2\pi i / 3}.
\]
\[
n = 1 + \log_i \omega = \frac{7}{3} + 4m, \quad n = \frac{11}{3} + 4m, \quad m \in \mathbb{Z}.
\]

\subsection{Quadratic with Impower Coefficients}
\[
(\text{Im}_n(a))^2 + \text{Im}_p(a) \cdot \text{Im}_n(a) + c = 0.
\]
\[
a^{2-2n} + a^{1-p} a^{1-n} + c = a^{2-2n} + a^{2-p-n} + c = 0.
\]
Multiply by \(a^{p+2n-2}\):
\[
a^p + a^n + c a^{p+2n-2} = 0.
\]
Let \(z = a^n\):
\[
c a^{p-2} z^2 + z + a^p = 0, \quad z = \frac{-1 \pm \sqrt{1 - 4 c a^{2p-2}}}{2 c a^{p-2}}.
\]
\[
n = \log_a \left( \frac{-1 \pm \sqrt{1 - 4 c a^{2p-2}}}{2 c a^{p-2}} \right).
\]

\section{Sequences and Series}
The sequence \(\text{Im}_{k+m}(a)\), \(m = 0, 1, 2, \dots\), is:
\[
s_m = a^{1-k-m}.
\]
\subsection{Geometric Sequence}
The ratio:
\[
\frac{s_{m+1}}{s_m} = \frac{a^{1-k-(m+1)}}{a^{1-k-m}} = a^{-1}.
\]
Thus, it is geometric with first term \(a^{1-k}\) and ratio \(\frac{1}{a}\).

\subsection{Series}
The infinite series is:
\[
S = \sum_{m=0}^\infty \text{Im}_{k+m}(a) = \sum_{m=0}^\infty a^{1-k-m}.
\]
Converges if \(\left| \frac{1}{a} \right| < 1\), i.e., \(|a| > 1\):
\[
S = \frac{a^{1-k}}{1 - a^{-1}} = \frac{a^{2-k}}{a - 1}.
\]
For \(a = i\), \(|i| = 1\), the series oscillates:
\[
1 - i - 1 + i + \cdots.
\]

\section{Applications}
For \(a = i\), \(\text{Im}_n(i) = i^{1-n} = (-i)^{n-1}\) cycles with period 4, modeling phase shifts in:
\begin{itemize}
    \item \textbf{Signal Processing}: Phase adjustments in Fourier transforms.
    \item \textbf{Quantum Mechanics}: Fractional rotations in qubit systems.
\end{itemize}
Impowers simplify iterative division processes in numerical methods or dynamical systems.

\section{Conclusion}
Impowers, while a reparameterization of exponentiation, offer a division-centric perspective that highlights inverse operations. Their algebraic rules, polynomial equations, and geometric sequences provide a cohesive framework with potential pedagogical and applied value, particularly for complex numbers in oscillatory systems.

\end{document}